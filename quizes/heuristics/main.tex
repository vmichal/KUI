\documentclass[twoside]{article}
\usepackage[a4paper]{geometry}
\geometry{verbose,tmargin=2.5cm,bmargin=2cm,lmargin=2cm,rmargin=2cm}
\usepackage{fancyhdr}
\pagestyle{fancy}

% nastavení pisma a češtiny
\usepackage{lmodern}
\usepackage[T1]{fontenc}
\usepackage[utf8]{inputenc}
\usepackage[czech]{babel}

% odkazy
\usepackage{url}

% vícesloupcové tabulky
\usepackage{multirow}
\usepackage{amssymb}
\usepackage{bbold}
\usepackage{amsmath}
\usepackage{commath}

% vnořené popisky obrázků
\usepackage{subcaption}

% automatická konverze EPS 
\usepackage{graphicx} 
\usepackage{epstopdf}
\epstopdfsetup{update}

\graphicspath{{./images}}

% odkazy a záložky
\usepackage[unicode=true, bookmarks=true,bookmarksnumbered=true,
bookmarksopen=false, breaklinks=false,pdfborder={0 0 0},
pdfpagemode=UseNone,backref=false,colorlinks=true] {hyperref}

% Poznámky při překladu
\usepackage{xkeyval}	% Inline todonotes
\usepackage[textsize = footnotesize]{todonotes}
\presetkeys{todonotes}{inline}{}

% enumerate zacina s pismenem
\renewcommand{\theenumi}{\alph{enumi}}

% smaz aktualni page layout
\fancyhf{}
% zahlavi
\usepackage{titling}
\fancyhf[HC]{\thetitle}
\fancyhf[HLE,HRO]{\theauthor}
\fancyhf[HRE,HLO]{\today}
 %zapati
\fancyhf[FLE,FRO]{\thepage}

% údaje o autorovi
\title{KUI kvíz 3 - Heuristika}
\author{Vojtěch Michal}
\date{\today}

\begin{document}

\maketitle


\section{Zadání}
Tu: \url{https://cw.fel.cvut.cz/wiki/_media/courses/b3b33kui/cviceni/program_po_tydnech/heuristics_ct_4mar_p103_cz.pdf}

Uvazujte graf v zadání, A je start, G je cıl. Ceny za prechod mezi uzly jsou vyznaceny u hran. Prechod
je mozny obema smery.
Dokoncete nıze uvedenou heuristickou funkci $h$. Vsechny jejı hodnoty jsou jiz dane, krome $h(B)$.

\begin{table}
	\centering
	\begin{tabular}{c|c|c|c|c|c|c|c|}
		Node & A & B & C & D & E & F & G \\ 
		\hline
		$h(\text{Node})$ & 10 & ? & 8 & 7 & 1 & 4 & 0
	\end{tabular}
\end{table}

\begin{itemize}
	\item Pro jake hodnoty $h(B)$ je heuristika h prıpustna (admissible) ?
	\item Jake hodnoty $h(B)$ cinı heuristiku h konzistentnı (consistent)?
\end{itemize}

\section{Přípustná heuristika}

Přípustná heuristika nikdy nenadsadí cenu dosažení bodu, tedy pro libovolný jiný vrchol X bude platit
\begin{equation*}
	h(B) \le d(X, B),
\end{equation*}
kde $d(X, B)$ je optimální vzdálenost vrcholů X a B. Protože minimum funkce $d(X, B)$ je $d(A, B) = 1$ pro vrchol $X = A$,
musí být heuristika menší nebo rovna této vzdálenosti. Proto
\begin{equation*}
	h(B) \le 1.
\end{equation*}

\section{Konzistentní heuristika}

\begin{thebibliography}{9}

\bibitem{rickroll}
	, \emph{Practical approach to Turing's test} \url{https://www.youtube.com/watch?v=dQw4w9WgXcQ}
\end{thebibliography}












\end{document}

