\documentclass[twoside]{article}
\usepackage[a4paper]{geometry}
\geometry{verbose,tmargin=2.5cm,bmargin=2cm,lmargin=2cm,rmargin=2cm}
\usepackage{fancyhdr}
\pagestyle{fancy}

% nastavení pisma a češtiny
\usepackage{lmodern}
\usepackage[T1]{fontenc}
\usepackage[utf8]{inputenc}
\usepackage[czech]{babel}

% odkazy
\usepackage{url}

% vícesloupcové tabulky
\usepackage{multirow}
\usepackage{amssymb}
\usepackage{bbold}
\usepackage{amsmath}
\usepackage{commath}

% vnořené popisky obrázků
\usepackage{subcaption}

% automatická konverze EPS 
\usepackage{graphicx} 
\usepackage{epstopdf}
\epstopdfsetup{update}

\graphicspath{{./images}}

% odkazy a záložky
\usepackage[unicode=true, bookmarks=true,bookmarksnumbered=true,
bookmarksopen=false, breaklinks=false,pdfborder={0 0 0},
pdfpagemode=UseNone,backref=false,colorlinks=true] {hyperref}

% Poznámky při překladu
\usepackage{xkeyval}	% Inline todonotes
\usepackage[textsize = footnotesize]{todonotes}
\presetkeys{todonotes}{inline}{}

% enumerate zacina s pismenem
\renewcommand{\theenumi}{\alph{enumi}}

% smaz aktualni page layout
\fancyhf{}
% zahlavi
\usepackage{titling}
\fancyhf[HC]{\thetitle}
\fancyhf[HLE,HRO]{\theauthor}
\fancyhf[HRE,HLO]{\today}
 %zapati
\fancyhf[FLE,FRO]{\thepage}

% údaje o autorovi
\title{KUI kvíz 2 - Letová hladina}
\author{Vojtěch Michal}
\date{\today}

\begin{document}

\maketitle


\section{Zadání}
Let mezi mesty A a B je dlouhy 1500 kilometruu. Mate pripravit optimalnı letovy plan s ohledem na
spotrebu paliva pro letadlo mezi temito mesty. K dispozici mate ctyri letove hladiny a tri operace -
up, keep level, down. Operacı up stoupa letadlo mezi hladinami a posune se o jednotku letu, keep
level pokracuje na dane hladine v letu a down naopak snizuje letovou hladinu za soucasneho posunu o
jednotku letu. Spotreba paliva se pocıta na 50 kilometru. Let na prvnı hladine stojı $k_1 = 14$ jednotek paliva,
let na druhe letove hladine stojı $k_2 = 13$ jednotek paliva, let na tretı letove hladine stojı $k_3 = 11$ jednotek paliva
a let na poslednı hladine stojı $k_4 = 9$ jednotek paliva. Zmena mezi prvnı a druhou letovou hladinou stojı $p_1 = 19$
jednotek paliva, zmena mezi druhou a tretı hladinou stojı $p_2 = 21$ jednotek paliva a zmena mezi poslednımi
dvema hladinami stojı $p_3 = 23$ jednotek paliva.

\begin{itemize}
	\item Jaky je tedy optimalni letovy plan pro tento let?
	\item Upravte ceny stoupani a klesani tak, aby bylo vyhodnejsı letet nejakou jinou letovou hladinou, napiste kterou a popiste letovy plan.
\end{itemize}

\section{Analýza}
Let sestává z $\frac{1500 \text{km} }{50 \text{km} } = 30$ dílků cesty. Protože jsou ceny letu na dílčích hladinách
konstantní, nemá smysl za letu měnit hladinu. Optimální je vybrat jednu hladinu na začátku a na konci z ní sestoupit.
Celková cena letu $N_h$ při použití letové hladiny $h$ se seskládá dle rovnice
\begin{equation}
	\label{eq:cena}
	N_h = \sum_{i=1}^{h-1} 2 \cdot p_i + (\underbrace{30 - 2\cdot (h-1)}_{\text{délka letu v hladině $h$}}) \cdot \underbrace{k_i}_{\text{cena letu v hladině $h$}},
\end{equation}
neboť vystoupáme-li na počátku do hladiny $h$, provedeme $h-1$ přechodů s cenou $p_i$ a na konci letu je musíme provést znovu k sestoupení na zem.
Ve střední části letu letíme na stejné hladině s cenou $k_i$ na dílek.

\section{Optimální plán letu}
Optimální plán letu nalezneme dosazením do vzorce \eqref{eq:cena} a nalezením nejmenšího N.
\begin{equation}
	\begin{split}
		N_1 &= 0 + 30 \cdot 14 = 420 \\
		N_2 &= 2 \cdot (19) + 28 \cdot 13 = 402 \\
		N_3 &= 2 \cdot (19 + 21) + 26 \cdot 11 = 366 \\
		N_4 &= 2 \cdot (19 + 21 + 23) + 24 \cdot 9 = 342 \\
	\end{split}
\end{equation}
Protože hledáme minimální $N$, je nejlevnější zjevně hladina $h = 4$ s cenou $N_4 = 342$ jednotek paliva.

\section{Úprava cen přechodů mezi hladinami}
Nejsnazší je vynutit let v první hladině, jehož cena je $N_1 = 420$. Pakliže bychom nastavili například $p_1$ (cenu přechodu z první hladiny na druhou)
na $\frac{N_1}{2}$, poté by každá další letová hladina byla zjevně dražší a optimální let by proběhl na hladině $h = 1$. To je vidět při dosažení do rovnice \eqref{eq:cena} pro $h>1$:
\begin{equation*}
	N_h = 2\cdot \frac{N_1}{2} + \underbrace{\sum_{i=2}^{h-1} 2 \cdot p_i + (30 - 2\cdot (h-1) \cdot {k_i}}_{\ge 0} \ge N_1.
\end{equation*}

Nutno poznamenat, že volba ceny přechodu $p_1$ rozhodně není nejmenší možná. Protože jsou všechny ceny nezáporné (dokonce kladné), není možné
odečítat cenu z již naakumulované sumy. Proto není potřeba provést tak výrazné zvětšení ceny přechodu $p_1$.
\begin{thebibliography}{9}

\bibitem{rickroll}
	, \emph{Ultimate guide how to program the universe} \url{https://www.youtube.com/watch?v=dQw4w9WgXcQ}
\end{thebibliography}












\end{document}

